\documentclass{muratcan_cv}

\setname{Tristan}{Touazi}
\setposition{Senior DevOps \textbar{} Founding Engineer}
\setaddress{142 chemin des Ouides 13760 Saint-Cannat}
\setmobile{06 09 46 37 47}
\setmail{tristan.touazi@gmail.com}
% Example: https://www.linkedin.com/in/votre-handle
\setlinkedinaccount{}
\setgithubaccount{https://github.com/TristanTouazi}
\setthemecolor{MidnightBlue}

\begin{document}
\headerview
\vspace{1.5ex}

\section{Profil}
\explanationdetail{
    Développeur expérimenté, je suis capable de concevoir et améliorer des programmes informatiques. Force de proposition avec un bon esprit d'analyse, je partage mon expertise technique et apporte des solutions efficaces. De plus, je sais rédiger des notices techniques et des guides pour les utilisateurs.
}
\vspace{2ex}

\section{Compétences}
\createskills{
    \createskill{Langages de programmation}{Python \cpshalf React \cpshalf Go \cpshalf C++ \cpshalf C\# \cpshalf TypeScript \cpshalf Java \cpshalf Bash \cpshalf SQL \cpshalf Android \cpshalf Lua},
    \createskill{Cloud \& DevOps}{GitHub Actions \cpshalf AWS (EC2, S3, DynamoDB, Lambda) \cpshalf GCP \cpshalf GitLab CI \cpshalf Docker \cpshalf K8s},
    \createskill{Outils \& moteurs}{IA \& LLM \cpshalf Unity \cpshalf UE5 \cpshalf Figma \cpshalf Visual Studio \cpshalf Confluence \cpshalf Jira \cpshalf Linear \cpshalf Datadog},
    \createskill{Pratiques}{CI/CD \cpshalf Automatisation de tests \cpshalf Gestion de produit \cpshalf Leadership d'équipe \cpshalf Agile/Scrum}
}
\vspace{-3mm}

\begin{comment}
\section{Experience}
%
\datedexperience{Sony Interactive Entertainment (PlayStation)}{Feb 2023 - Jul 2025}
\explanation{DevOps Engineer / Autotests Infrastructure Lead}{San Francisco (Remote)}
\explanationdetail{
    \smallskip
    \coloredbullet\ %
    Built from scratch the \textbf{CI/CD and autotest infrastructure} for Sony PlayStation emulators, used in the \textbf{Classics Catalog (PlayStation Plus)} on PS4 and PS5.

    \smallskip
    \coloredbullet\ %
    Automated testing of \textbf{5,000+ retro games} on various \textbf{console devkits and OS environments}, running continuously via \textbf{scheduled jobs and commit-based triggers}, using \textbf{Python}, \textbf{GitHub Actions}, \textbf{AWS}, and \textbf{Google/Slack/Linear APIs}.

    \smallskip
    \coloredbullet\ %
    Ensured quality and \textbf{non-regression testing} for \textbf{50+ PlayStation Classics}, some played by up to \textbf{1M players} on PS4/PS5, including \textit{Resident Evil}, \textit{Tomb Raider Legend}, \textit{Tekken 2} and \textit{Legend of Dragoon}.

    \smallskip
    \coloredbullet\ %
    Principal maintainer and developer of \textbf{CI/CD pipelines} used by \textbf{15+ engineers}; led internal \textbf{QA processes} and provided consulting for Sony's QA team of \textbf{60+ Sony manual testers}.

    \smallskip
    \coloredbullet\ %
    Through \textbf{Implicit Conversions Inc.}, contributed to \textbf{porting and publishing retro titles} on PlayStation, Switch, and PC - including \textit{Micro Mages} and \textit{Fear Effect}.
}
%
\datedexperience{Jam.gg (YCombinator S20)}{Mar 2020 - Feb 2023}
\explanation{Founding Engineer / Product Manager}{San Francisco (Remote)}
\explanationdetail{
    \smallskip
    \coloredbullet\ %
    Part of the cofounding team of a \textbf{SaaS cloud gaming platform} (ex-Piepacker) incubated by \textbf{YCombinator}; scaled the product to \textbf{2M users} and \textbf{500K MAU}, grew the company to \textbf{60+ employees} and raised a \textbf{\$12M Series A}.

    \smallskip
    \coloredbullet\ %
    Core contributor to the \textbf{platform architecture} and feature implementation using \textbf{Go}, \textbf{TypeScript}, and \textbf{React} across \textbf{backend, frontend, and cloud} systems (Firebase, GCP).

    \smallskip
    \coloredbullet\ %
    Created and led the \textbf{product} and \textbf{QA teams}; managed \textbf{recruitment of the first 30 hires} and shaped the \textbf{company culture}, overseeing agile practices and cross-team coordination.

    \smallskip
    \coloredbullet\ %
    Elaborated and maintained the \textbf{product roadmap}; led \textbf{specs and delivery} for major features such as Discord/Twitch integration, browser-based Twitch streaming, and a game upload and catalog system.
}
%
\datedexperience{Atos}{Sep 2018 - Jul 2020}
\explanation{Innovation Project Manager / Backend Engineer}{Aix-en-Provence, France}
\explanationdetail{
    \smallskip
    \coloredbullet\ %
    Led a team of \textbf{6 developers} on \textbf{AI, IoT, and predictive maintenance} projects, designing custom \textbf{ESP32-based sensors} and building dashboards following \textbf{open data} standards, with \textbf{Microsoft Azure}.

    \smallskip
    \coloredbullet\ %
    Developed \textbf{Java} backend for a secure \textbf{geolocation tool} used by the \textbf{French Army}; also built a \textbf{Grafana dashboard} to visualize code quality for a team of \textbf{50+ engineers}.
}
%
\datedexperience{Airbus Helicopters}{Oct 2017 - Sep 2018}
\explanation{VR/AR Engineer}{Marignane, France}
\explanationdetail{
    \smallskip
    \coloredbullet\ %
    Built \textbf{immersive simulators} in \textbf{Unity (C\#)} for \textbf{HTC Vive} and \textbf{Microsoft HoloLens} to train helicopter technicians.

    \smallskip
    \coloredbullet\ %
    Developed a \textbf{Python/XML tool} for \textbf{Dassault} to generate and maintain technical documentation for helicopters.
}
%
\datedexperience{CNRS (French National Centre for Scientific Research)}{May 2016 - Jul 2016}
\explanation{Bioinformatics Engineer Intern}{Marseille, France}
\explanationdetail{
    \coloredbullet\ %
    Extended \textbf{VisualTE}, a Java/Swing tool for studying the evolution of genetic transposable elements, by adding modules for \textbf{2D/3D structure analysis}, \textbf{multiple alignments}, and \textbf{phylogenetic visualization}.
}
\vspace{1ex}
\section{Education}
\datedexperience{Master's Degree in Software Engineering - AI \& Machine Learning}{2016 - 2019}
\explanation{Ingesup, Engineering School}{Aix-en-Provence, France}
\datedexperience{Bachelor's Degree in Cell \& Molecular Biology}{2013 - 2016}
\explanation{Aix-Marseille University, Faculty of Science}{Marseille, France}
\end{comment}

\section{Expérience}
% Airbus Helicopters
\datedexperience{Airbus Helicopters}{2014 — 2019}
\explanation{Technicien Aéronautique}{CDD — Marignane}
\explanationdetail{
    \smallskip
    \coloredbullet\\ %
    Remise en état des équipements défectueux : changement de pièces, reprogrammation du système.

    \smallskip
    \coloredbullet\\ %
    Diagnostic de panne : localisation de la panne, évaluation des travaux de réparation à faire.

    \smallskip
    \coloredbullet\\ %
    Travaux d'entretien et de maintenance préventive sur des équipements électriques et électroniques.

    \smallskip
    \coloredbullet\\ %
    Participation aux programmes d'amélioration continue visant à optimiser les processus de maintenance.

    \smallskip
    \coloredbullet\\ %
    Actualisation de la documentation existante afin d'améliorer la fiabilité et la pertinence des process.
}
% Technique Atome
\datedexperience{Technique Atome}{2020 — 2022}
\explanation{Développeur Python}{CDD — Aix-en-Provence}
\explanationdetail{
    \smallskip
    \coloredbullet\\ %
    Analyse des besoins clients, étude de la faisabilité de leur projet et établissement du cahier des charges.

    \smallskip
    \coloredbullet\\ %
    Participation à des projets de transformation digitale en collaborant étroitement avec des équipes multidisciplinaires.

    \smallskip
    \coloredbullet\\ %
    Amélioration de logiciels existants en effectuant la mise à niveau des interfaces et l'adaptation à un nouveau matériel.

    \smallskip
    \coloredbullet\\ %
    Modification de bases de données existantes pour répondre aux nouveaux besoins des utilisateurs.

    \smallskip
    \coloredbullet\\ %
    Conception et programmation d'applications et de solutions logicielles dans le respect du cahier des charges.
}
% Freelance associatif
\datedexperience{Freelance associatif}{2022 — 2025}
\explanation{Développeur / Projets associatifs}{}
\begin{comment}
\explanationdetail{
    \smallskip
    \coloredbullet\\ %
    Analyse des besoins clients, étude de la faisabilité de leur projet et établissement du cahier des charges.

    \smallskip
    \coloredbullet\\ %
    Participation à des projets de transformation digitale en collaborant étroitement avec des équipes multidisciplinaires.

    \smallskip
    \coloredbullet\\ %
    Amélioration de logiciels existants en effectuant la mise à niveau des interfaces et l'adaptation à un nouveau matériel.

    \smallskip
    \coloredbullet\\ %
    Modification de bases de données existantes pour répondre aux nouveaux besoins des utilisateurs.

    \smallskip
    \coloredbullet\\ %
    Conception et programmation d'applications et de solutions logicielles dans le respect du cahier des charges.
}
\end{comment}

% Contenu associatif
\explanationdetail{
    \smallskip
    \coloredbullet\\ %
    Animation et encadrement de bénévoles; organisation d'ateliers d'initiation au numérique et d'événements locaux.

    \smallskip
    \coloredbullet\\ %
    Mise en place d'outils collaboratifs (wiki, gestion de projets, messagerie) et bonnes pratiques de contribution.

    \smallskip
    \coloredbullet\\ %
    Communication associative: rédaction de supports, publications sur les réseaux sociaux et relations partenaires.

    \smallskip
    \coloredbullet\\ %
    Recherche de financements et suivi des subventions; élaboration de budgets et bilans d'activité.

    \smallskip
    \coloredbullet\\ %
    Veille, inclusion et sensibilisation au numérique responsable; accompagnement des publics et accessibilité.
}

\vspace{1ex}
\section{Formation}
\datedexperience{DUT : Génie électrique et informatique industrielle}{}
\explanation{IUT de Salon de Provence}{Salon-de-Provence}
\datedexperience{Licence professionnelle : Intégration des systèmes électriques en aéronautique}{}
\explanation{IUT de Salon de Provence}{Salon-de-Provence}
\datedexperience{Master 2 : Mastère Expert en développement mobile \& IOT}{}
\explanation{YNOV}{Aix-en-Provence}

\end{document}
